%-----------------------------------------------------------------
%-----------------------------------------------------------------
\begin{plSection}{Feferman}
\label{sec:Feferman}
\citeAuthorYearTitle{Feferman:1998:LightOfLogic,Feferman:2000:ConstructivePredicativeClassicalAnalysis}

%-----------------------------------------------------------------
\begin{plSection}{In the light of logic}
\label{sec:In_the_light_of_logic}

Fourteen collected essays about logic and foundations of 
math. \citeAuthorYearTitle{Feferman:1998:LightOfLogic}
Not strictly about constructivism, but many references to it.
Probably more focused on finitism.

%-----------------------------------------------------------------
\begin{plSection}{Deciding the undecidable}
\label{sec:Deciding_the_undecidable}

\citeAuthorYearTitle[ch.~1 ``Deciding the undecidable'']{Feferman:1998:LightOfLogic}

Target general audience.

Background on $3$ of Hilbert's problems
and effect of \Godel incompleteness theorems.

Some discussion of constructive approaches, 
predicative/impredicative, finitary/infinitary.

Metamathematical proof theory as result.

Question of (non)existence 
'genuine absolutely undecidable problems'.

Conclusion mentions recent progress
in purely finitary approaches~\cite[ch.~10]{Feferman:1998:LightOfLogic}
and 
use of 'proof-theoretically very weak systems'
as basis for 'an enormous amount of scientifically applicable
mathematics.'~\cite[ch.~14]{Feferman:1998:LightOfLogic}
(!! what I'm looking for !!, ie, 
what's the 'minimal' mathematical structure needed to
solve the real world problems I'm interested in?)

\end{plSection}%{Deciding the undecidable}
%-----------------------------------------------------------------
\begin{plSection}{Is Cantor Necessary?}
\label{sec:Is_Cantor_Necessary}

\citeAuthorYearTitle{Feferman:1989:IsCantorNecessary}
\citeAuthorYearTitle[ch.~2 ``Is Cantor necessary?'']{Feferman:1998:LightOfLogic}

Cantor diagonal argument:
\hfill\break
$\OneToOne$ map between binary sequences ($2^{\Space{N}}$)
and $[0,1]\subset\Space{R}$
Does that need a proof?
What about finite sequences vs trailing $1$s? 
\hfill\break
Then assume existence of a $\OneToOne$  
map between $2^{\Space{N}}$s 
and $\Space{N}$,
without any explicit construction.
Construct an explicit missing sequence 
for any candidate $\OneToOne$ map,
by flipping the $i$th bit in the $i$th sequence,
thus a contradiction?
So $\mathfrak{c} = \card(2^{\Space{N}}) 
= \card(\Space{R})$ must be greater than 
$\aleph_0 = \card(\Space{N})$. 
\hfill\break
Claim this can be extended to any set $\Set{S}$,
proving 
$\card(\Set{S}) < \card(2^{\Set{S}})$,
including 
$\card(\Space{R}) < \card(2^{\Space{R}})$?
Doesn't seem obvious.
\hfill\break
How does all this play with 
the Hilbert hotel paradox? \citeAuthorTitle{wiki:HilbertHotel}
(Problem with Hilbert hotel is 
always space for new customer, if everybody moves one room up,
but infinite time required to make that space.)

Question $1$ for Cantor:
\hfill\break
Are there any cardinals other than 
$\card(\Space{N})$ and 
$\card(\Space{R})$?

Question $2$ for Cantor (Continuum hypothesis aka CH):
\hfill\break
Are there any cardinals between
$\card(\Space{N})$ and 
$\card(\Space{R})$?

Implicit of axiom of choice (AOC) used to 'prove' 
well ordering (WO) of cardinals.
('Cardinals' not clearly defined.
Should it be equivalence classes of sets where
equivalence means 'there exists'/'we can construct'
a $\OneToOne$ mapping.)
\hfill\break
Well ordering used to show 'scale' of infinite cardinals
$\aleph_0, \aleph_1, \ldots$. (Is this countable?)
\hfill\break
Continuum hypothesis: 
$\mathfrak{c} = \card(2^{\Space{N}}) 
> \card(\Space{R}) = \aleph_1$
(and there are no cardinalities in between).

Zermelo axioms of set theory, including AC.
Start from axioms defining initial universe:
empty set and $\Space{N}$.
Then set operations including predicated subsets,
'finessing' the contradiction of the
cardinality of the set of all cardinalities,
set of all sets,
sets of all sets not containing themselves, 
\ldots.
\hfill\break
'\ldots more fuel for criticism \ldots.'
Weyl and Skolem improvements by basing it on 1st order logic.
(But doesn't a first order language require sets,
eg, a set of atomic symbols?)

P.~$60$ $2$nd paragraph (Skolem's paradox):
\hfill\break
What's the difference between $\Set{A} \sim \PowerSet{\Set{A}}$ 
($\card(\Set{A})=\card(\Set{\PowerSet{\Set{A}}})$
 'externally' vs 'internally'?

\textsb{\Godel's incompleteness theorems:}
don't follow completely.

\textsb{Postscript on second-order logic:}
``second-order logic cannot be axiomatized effectively.''
Don't follow much of argument.

\textsb{Elimination of the law of the excluded middle:}
\hfill\break
LEM implies (for Hilbert) completed infinity (?).
\hfill\break
Heyting arithmetic (HA) excludes LEM;
any true statement in PA (Peano arithmetic) 
(and many other arithmetic systems)
can be translated into
a true statement in HA.
\hfill\break
This (somehow) implies that a finite consistency proof
of PA would not eliminate completed infinity.

\textsb{The elusiveness of Cantor's continuum problem:}
\hfill\break
Hilbert's program can't solve the continuum problem.


\textsb{New axioms?}
\hfill\break
Dubiousness of large cardinals.
\hfill\break
Freiling axiom of 
symmetry (AX) \citeAuthorYearTitle{Freiling:1986:Symmetry,wiki:FreilingsAxiomOfSymmetry}
'disproves' CH.

\end{plSection}%{Is Cantor Necessary?}
%-----------------------------------------------------------------
\begin{plSection}{The logic of mathematical discovery 
versus the logical structure of mathematics}
\label{sec:logic_of_mathematical_discovery}

\citeAuthorYearTitle[ch.~3 ``The logic of mathematical discovery versus
the logical structure of mathematics'']{Feferman:1998:LightOfLogic}

Reviews work of Lakatos \citeAuthorYearTitle{Lakatos:1976:Proofs,Lakatos:1978:MSE},
a fairly dubious 'theory' of how mathematics research is done,
neither empirical (no significant evidence, just anecdotes),
nor mathematically/logically rigorous.

Brief comparison to 
P\'{o}lya~\cite[ch~3,sec~7]{Feferman:1998:LightOfLogic},
which sounds much more useful. \citeAuthorYearTitle{Polya:1957:SolveIt,
Polya:1965:MathDiscovery,Polya:1968:PlausibleReasoning}
``\ldots P\'{o}lya \ldots concentrates on tactics and methods
for \textit{finding solutions} to problems and, 
to a lesser extent,
on \textit{finding proofs} of theorems.''
\hfill\break
``\ldots professional mathematicians might want \ldots
a continuation \ldots which concentrated on \ldots
finding difficult proofs.''

Neither Lakatos or P\'{o}lya ``\ldots attempted to deal with 
\ldots
finding the technical but general concepts that 
help organize masses of material and make difficult 
proofs understandable.''

\end{plSection}%{The logic of mathematical discovery versus the logical structure of mathematics}
%-----------------------------------------------------------------
\begin{plSection}{Foundational ways}
\label{sec:Foundational_ways}

\citeAuthorYearTitle[ch~4 ``Foundational ways'']{Feferman:1998:LightOfLogic}

$20$th century crisis in mathematics and grand solutions:
\begin{plQuote}{\citeAuthorYearTitle[ch~4 ``Foundational ways'']{Feferman:1998:LightOfLogic}}{}
\ldots\ logicism, formalism, platonism, and constructivism \ldots\
are all rather tired looking now, if not suffering from
senescence and still more basic ills. \ldots
most mathematicians have given up worrying \dots .
If asked, say they are formalists, or that
Zermelo-Fraenkel meets their needs, or whatever.
\end{plQuote}

Argument against category theory as alternative 
foundation in \citeAuthorYearTitle{Feferman:1977:CategoricalFoundations}.

Critics of logical foundations argue ``that mathematics
is more reliable than any of the \ldots schemes which have been
[proposed] to 'secure' it.''

\cite[ch~4, p~99]{Feferman:1998:LightOfLogic} has
Scott models for (self-referential) lambda 
calculus. \citeAuthorYearTitle{Scott:1972:ContinuousLattices,
Scott:1976:DataTypesAsLattices}

Survey of models for self-referential theories in 
\citeAuthorYearTitle{Feferman:1984:TypeFreeTheoriesI}.

\end{plSection}%{Foundational ways}
%-----------------------------------------------------------------
\begin{plSection}{Working foundations 1991}
\label{sec:Working_foundations_1991}

\citeAuthorYearTitle[ch~5 ``Working foundations 1991'']{
Feferman:1998:LightOfLogic}

Rougher version of \cite[ch~4]{Feferman:1998:LightOfLogic},
see notes in \cref{sec:Foundational_ways}.

Reference to MacLane \citeAuthorYearTitle{MacLane:1981:MathModels},
which has some relevance to motivating
parts of mathematics (and foundations)
from applications.

\end{plSection}%{Working foundations 1991}
%-----------------------------------------------------------------
\begin{plSection}{Reflective expansion of concepts and principles}

Mentions, but does not explain, the difference 
in constructive mathematics between
``the idea of a function as given by a \textit{rule}
\ldots [and] \ldots the idea of a set as given by a 
\textit{defining property}. 
\citeAuthorYearTitle{Feferman:1979:ConstructiveFunctionsClasses}

\end{plSection}%{Reflective expansion of concepts and principles}
%-----------------------------------------------------------------
\begin{plSection}{\Godel's life and work}
\label{sec:Godels_life_and_work}

\citeAuthorYearTitle[ch~6 ``\Godel's life and work'']{
Feferman:1998:LightOfLogic}

\end{plSection}%{\Godel's life and work}
%-----------------------------------------------------------------
\begin{plSection}{Kurt \Godel: conviction and caution}
\label{sec:Kurt_Godel_conviction_and_caution}

Examines admittedly limited and contradictory evidence
of \Godel's platonist/realist/objectivist view of
mathematics: abstract mathematical entities exist outside 
of anybody thinking about them;
undecidable propositions are in fact true or false, we just don't
know (?).
Also considers whether \Godel concealed these views and 
why.
 \citeAuthorYearTitle[ch~7 ``Kurt \Godel: conviction and caution'']{
 Feferman:1998:LightOfLogic}

Incompleteness theorem uses self-referential statement
equivalent to [this statement cannot be 
proved].~\cite[p~156]{Feferman:1998:LightOfLogic}

Self-referential paradoxes 'resolved' by 
``\ldots 'false statement in [language] B' cannot be expressed in
B, and so [is a] \ldots statement \ldots in some other 
language''.~\cite[p~157]{Feferman:1998:LightOfLogic}

\Godel refers to 
Tarski 
\citeAuthorYearTitle{
Tarski1944:SemanticTruth,
Tarski:1983:LogicEtc} 
for justification in 1965 version 
of this work. \citeAuthorYearTitle{Godel:1986:CollectedWorksI}
 
(Can we construct a language that can express 
or evaluate the truth of 
its own statements? 
Is non-halting evaluation a way out of paradoxes?)

``Thus if truth of number theory \textit{were}
definable within itself, one could find a precise version of the
Liar statement, giving a contradiction.
It follows that truth is not so definable.
But provability in the system \textit{is} definable,
so the notions of provability and truth must be distinct.
In particular, if all provable sentences are true,
there must be true non-provable sentences.
The self-referential construction applied to provability
(which \textit{is} definable) instead of truth then 
leads to a specific example of an undecidable 
sentence.''~\cite[p~159]{Feferman:1998:LightOfLogic}

(Is 'provability' the same as non-halting truth evaluation?)

\end{plSection}%{Kurt \Godel: conviction and caution}
%-----------------------------------------------------------------
\begin{plSection}{Introductory note to Kurt \Godel's 1933 lecture}
\label{sec:Introductory_note_Godel_1933}

\citeAuthorYearTitle[ch~8 ``Introductory note to Kurt \Godel's 1933 lecture'']{Feferman:1998:LightOfLogic}

See also \citeAuthorYearTitle{Godel:1995:CollectedWorksIII}.

\end{plSection}%{Introductory note to Kurt \Godel's 1933 lecture}
%-----------------------------------------------------------------
\begin{plSection}{What does logic have to tell us about mathematical proofs?}
\label{sec:What_does_logic_tell_us_about_proofs}

\citeAuthorYearTitle[ch~9 ``What does logic have to tell us about mathematical proofs?'']{Feferman:1998:LightOfLogic}

``One part of the analogy with physics is not so apt.
The experimental method provides highly sophisticated, refined
means to test physical theories {\ldots}.
We have no such tests of logical theories.
Rather it is primarily a matter of individual judgment
how well these square with ordinary
experience."~\cite[p~178]{Feferman:1998:LightOfLogic}

``\ldots the process of \textit{explicit definition,} which allows
us to expand our basic vocabulary systematically,
is necessary to keep formal representations of informal concepts
down to a manageable size. \ldots\ 
If everything were written out in terms of primitive notions,
mathematics would be 
unlearnable.''~\cite[p~180]{Feferman:1998:LightOfLogic}

Bishop's work can be formalized 
in Feferman's $T_0$ \citeAuthorYearTitle{Feferman:1975:ExplicitMathematics}
(a subsystem of Myhill's CST \citeAuthorYearTitle{Myhill:1975:ConstSetTheory} (?)).

\end{plSection}%{What does logic have to tell us about mathematical proofs?}
%-----------------------------------------------------------------
\begin{plSection}{What rests on what? The proof-theoretic analysis of mathematics}
\label{sec:What_rests_on_what}

\citeAuthorYearTitle[ch~10 ``What rests on what? 
The proof-theoretic analysis of mathematics'']{Feferman:1998:LightOfLogic}

``\Godel \ldots\ showed that PA [Peano arithmetic]
could be translated into the intuitionistic system HA of
Heyting's arithmetic,
which differs from PA on in omitting the 
Law of the Excluded Middle{\ldots}.''~\cite[190]{Feferman:1998:LightOfLogic}

\end{plSection}%{What rests on what? The proof-theoretic analysis of mathematics}
%-----------------------------------------------------------------
\begin{plSection}{\Godel's \textit{Dialectica} interpretation and its two-way-stretch}
\label{sec:Godels_Dialectica_interpretation}

\citeAuthorYearTitle[ch~11 ``\Godel's \textit{Dialectica} interpretation and its
two-way-stretch'']{Feferman:1998:LightOfLogic}

\end{plSection}%{\Godel's \textit{Dialectica} interpretation and its two-way-stretch}
%-----------------------------------------------------------------
\begin{plSection}{Infinity in mathematics: is Cantor necessary? (conclusion)}
\label{sec:Cantor_necessary_conclusion}

\citeAuthorYearTitle[ch~12 ``Infinity in mathematics: is Cantor necessary? (conclusion)'']
{Feferman:1998:LightOfLogic}

Classes of statements and formulas:
\begin{description}
\item[$\Pi^0_1$] statements of the form
$\forall x (F(n) = 0)$ where $F$ is primitive recursive.
\item[$\Sigma^0_1$] statements of the form
$\exists x (F(n) \neq 0)$ (negations of statement in $\Pi^0_1$).
\item[$\Pi^0_k$] sequence of $k$ alternating quantifiers,
starting with $\forall x$.
\textsb{(Why alternating?)}
\item[$\Sigma^0_k$] sequence of $k$ alternating quantifiers,
starting with $\exists x$.
\textsb{(Why alternating?)}
\item[$\Delta^0_k$] statements that are in both $\Pi^0_k$
and $\Sigma^0_k$. 
\textsb{(How is that possible?)}
\end{description}
Superscript $0$ is above classes refers to the fact that
quantified variables range over $x,y,\ldots \in \Space{N}$.
\textsb{(first order logic/arithmetic?)}
When we add variables $X,Y,\ldots \subseteq \Space{N}$,
then the classes are $\Pi^1_k, \Sigma^1_k, \Delta^1_k$.

Collecting all arities into a single class:
\begin{itemize}
  \item $\Pi^i_{\infty} \, = \, \bigcup_k \Pi^i_k$
  \item $\Sigma^i_{\infty} \, = \, \bigcup_k \Sigma^i_k$
  \item $\Delta^i_{\infty} \, = \, \bigcup_k \Delta^i_k$
\end{itemize}

\textit{Comprehension axiom schema:}
\begin{equation}
\tag{$\mathcal{F}{-}\text{CA}$}
\exists X \forall n \left[ (n \in X) \Leftrightarrow P(n) \right]
\end{equation}
where $\mathcal{F}$ is a class of formulas 
\textsb{(meaning $\Pi^0_k$ or $\Sigma^0_k$?)}
and $P \in \mathcal{F}$.

For formulas in $\mathcal{F}$ and its complement,
\begin{equation}
\tag{$\Delta_{\mathcal{F}}{-}\text{CA}$}
\forall n \left[ P(n) \Leftrightarrow Q(n) \right]
\Rightarrow 
\exists X \forall n \left[ (n \in X) \Leftrightarrow P(n) \right]
\end{equation}
where $P \in \mathcal{F}$ and $\sim Q \in \mathcal(F)$.
\textsb{
(Is ``$\sim Q$'' the same as ``$\lnot Q$'', ie, 
``$(\text{not}\, Q)$''?)}

\end{plSection}%{Infinity in mathematics: is Cantor necessary? (conclusion)}
%-----------------------------------------------------------------
\begin{plSection}{Weyl vindicated: \textit{Das Kontinuum} seventy years later}
\label{sec:Weyl_vindicated}

Moderate number of typos in this 
chapter. \citeAuthorYearTitle[ch~13 ``Weyl vindicated: \textit{Das Kontinuum} seventy years later'']{Feferman:1998:LightOfLogic}

Goal is to fix Weyl's program 
using 'finite', 'definitionist/predicative', 'intuitionist' ideas
to create a basis for scientifically applicable mathematics
(without uncountable infinities?).

Work on foundations of math \citeAuthorYearTitle{Weyl:1910:Definitionen}.
\textit{Das Kontinuum} \citeAuthorYearTitle{Weyl:1918:Kontinuum,Weyl:1987:Continuum}.

``Weyl's way out'' of Richard paradox/diagonal 
argument.~\cite[p 262]{Feferman:1998:LightOfLogic} 

Theory W (a conservative extension of PA),
with 'variable/flexible' finite types,
claims to formalize substantial parts of
classical and modern 
analysis.~\cite[ch~13 sec~8]{Feferman:1998:LightOfLogic}

\end{plSection}%{Weyl vindicated: \textit{Das Kontinuum} seventy years later}
%-----------------------------------------------------------------
\begin{plSection}{Why a little bit goes a long way:
logical foundations of scientifically applicable mathematics}
\label{sec:Why_a_little_bit_goes_a_long_way}

\citeAuthorYearTitle[ch~14 ``Why a little bit goes a long way:
logical foundations of scientifically applicable mathematics'']{Feferman:1998:LightOfLogic}

Quite a bit of overlap (system W) 
with~\cite[ch 13]{Feferman:1998:LightOfLogic}.
Not much additional detail about what, exactly,
is the range of ``scientifically applicable mathematics''
or how system W fulfills that.

\end{plSection}%{Why a little bit goes a long way: logical foundations of scientifically applicable mathematics}
%-----------------------------------------------------------------
\end{plSection}%{In the light of logic}
%-----------------------------------------------------------------
\end{plSection}%{Feferman}
%-----------------------------------------------------------------
 