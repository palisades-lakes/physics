\begin{plSection}{Mathematical Structures}

A mathematical structure~\cite{wiki:MathematicalStructure} 
consists of a few sets, and functions between those sets.
Types of structure:
\begin{description}
\item[Algebraic] 
Characterized by operations (functions of $2$ arguments, 
sometimes more) that have given properties, like commutativity:
$\left(+ \, a \, b \right) = \left(+ \, b \, a \right)$.

\item[Order and equivalence] 
A set plus a one or more relations. 
See \cref{sec:Relations}

\item[Topology] A primary set plus a family of 'open' subsets of the
primary set. The subsets have to obey certain properties which 
enable to be used to define what is 'connected' to what in the 
primary set.

\item[Metric] A set plus a \textit{distance} function,
mapping each pairs of elements to non-negative real number,
having certain properties.
Metric/distance induces a topology:
\[
\Set{B}(x_0,r) = \SetSpec{ x \in \Set{X} }{ 
\text{distance} \left(x_0 , x \right) < r }
\]
See \cref{sec:Metric-spaces}.

\item[Measure] A set plus a function that assigns a non-negative
real value to some subsets of the primary set. 
Examples are length, area, volume.

\item[Geometry] \textbf{TODO:}
\end{description}

For \glssymbol{RealNumbers}:
Order and Metric induce Topology.
Order and Algebraic structure lead to ordered field.
Algebraic structure and topology make Lie group.

There is usually a primary set, whose elements are the 'elements'
of the structure:

\begin{plExample}{Linear Space}{}
\bigskip
A \textit{linear space} $\Space{V}$ is:
\begin{itemize}
  \item a set of vectors $\Set{V}$,
  \item a field of scalars $\Space{F}$,
  \item a linear combination operation/function: 
\begin{equation}
\left( \linearCombination 
\, a_0 \, \Vector{v}_0 \, a_1 \, \Vector{v}_1 \right) \; 
= \; a_0*\Vector{v}_0 + a_1*\Vector{v}_1
\; \rightarrow \; \Vector{v}_2  \in \Set{V}
\end{equation}
for $\Vector{v}_0, \Vector{v}_1 \in \Set{V} $
and $a_0, a_1 \in \Space{F}$.
Linear combination is often defined in terms of
$2$ other operations:
scalar multiplication $a * \Vector{v} \in \Set{V}$,
and vector addition $\Vector{v}_0 + \Vector{v}_1 \in \Set{V}$
\end{itemize}
Usually the distinction between $\Set{V}$ and 
$\Space{V} = \left[ \Set{V}, \Space{F}, \linearCombination \right]$
is ignored;
I prefer to define in terms of $\linearCombination$ 
because it
naturally extends to $affineCombination$
and $\convexCombination$.
\end{plExample}

It's tempting to identify a mathematical structure with a class,
but that won't work, because you will need to support multiple
representations (dense and sparse vectors, \texttt{double}
and \texttt{BigFraction}) and the functions
operate on multiple sets (vectors and scalars)
each with multiple representations.
Interfaces also don't work, because 
operations are typically functions of more than one argument,
with multiple representations for each.

Implementation is easier with \textit{generic functions}
(aka ``multimethods'').
Or context object determining what $\left(+ \, a \, b \right)$
means for any acceptable implementation of $a$ and $b$.
\end{plSection}%{Mathematical Structures}
%-----------------------------------------------------------------
